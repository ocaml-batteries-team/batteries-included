
Some functions have exceedingly long names. Case in point :

\begin{verbatim}
let rec pretentious_drivel x0 f = function [] -> x0
  | x::xs -> pretentious_drivel (f x x0) f xs
\end{verbatim}

\begin{verbatim}
(*$T pretentious_drivel
  pretentious_drivel 1 (+) [4;5] = foo 1 (+) [4;5]
  ... pretentious_drivel of this and that...
*)
\end{verbatim}

The constraint that each statement must fit on one line does not play well with very long
function names. Furthermore, you \emph{known} which function is being tested, it's right there
is the header; no need to repeat it a dozen times. Instead, you can define an \emph{alias}, and
write equivalently:

\begin{verbatim}
(*$T pretentious_drivel as x
  x 1 (+) [4;5] = foo 1 (+) [4;5]
  ... x of this and that...
*)
\end{verbatim}

... thus saving many keystrokes, thereby contributing to the preservation of the
environment. More seriously, aliases have uses beyond just saving a few keystrokes, as we
will see in the next sections.