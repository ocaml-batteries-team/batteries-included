
When more specialised test pragmas are too restrictive, for instance if the test is too
complex to reasonably fit on one line, then one can use raw \ounit{} tests.

\begin{verbatim}
(*$R <header>
  <raw oUnit test>...
  ...
*)
\end{verbatim}

Here is a small example, with two tests stringed together:

\begin{verbatim}
(*$R foo
  let thing = foo  1 ( * )
  and li = [4;5] in
  assert_bool "something_witty" (thing li = 20);
  assert_bool "something_wittier" (foo 12 ( + ) [] = 12)
*)
\end{verbatim}

Note that if the first assertion fails, the second will not be executed; so stringing two
assertions in that mode is different in that respect from doing so under a \texttt{T} pragma, for
instance.

That said, raw tests should only be used as a last resort; for instance you don't
automatically get the source file and line number when the test fails. If \texttt{T} and \texttt{Q} do
not satisfy your needs, then it is \emph{probably} a hint that the test is a bit complex and,
maybe, belongs in a separate test suite rather than in the middle of the source code.