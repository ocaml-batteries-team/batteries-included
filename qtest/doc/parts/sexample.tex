
Say that you have a file \texttt{foo.ml}, which contains the implementation of your new, shiny
function \texttt{foo}.

\begin{verbatim}
let rec foo x0 f = function
  [] -> 0 | x::xs -> f x (foo x0 f xs)
\end{verbatim}


Maybe you don't feel confident about that code; or maybe you do, but you know that the
function might be re-implemented less trivially in the future and want to prevent
potential regressions. Or maybe you simply think unit tests are good practice anyway. In
either case, you feel that building a separate test suite for this would be overkill.
Using \qtest{}, you can immediately put simple unit tests in comments near \texttt{foo}, for
instance:

\begin{verbatim}
(*$T foo
  foo  0 ( + ) [1;2;3] = 6
  foo  0 ( * ) [1;2;3] = 0
  foo  1 ( * ) [4;5]   = 20
  foo 12 ( + ) []      = 12
*)
\end{verbatim}

the syntax is simple: \texttt{(*\$} introduces a \qtest{} "pragma", such as \texttt{T} in this case. \texttt{T}
is by far the most common and represents a "simple" unit test. \texttt{T} expects a "header",
which is most of the time simply the name of the function under test, here \texttt{foo}.
Following that, each line is a "statement", which must evaluate to true for the test to
pass. Furthermore, \texttt{foo} must appear in each statement.

Now, in order to execute those tests, you need to extract them; this is done with the
\qtest{} executable. The command

\Oconsole
\begin{verbatim}
$ qtest -o footest.ml extract foo.ml
Target file: `footest.ml'. Extraction : `foo.ml' Done.
\end{verbatim}

will create a file \texttt{footest.ml}; it's not terribly human-readable, but you can see that it
contains your tests as well as some \ounit{} boilerplate. Now you need to compile the
tests, for instance with \texttt{ocamlbuild}, and assuming \ounit{} was installed for \texttt{ocamlfind}.

\begin{verbatim}
$ ocamlbuild -cflags -warn-error,+26 -use-ocamlfind -package oUnit \
    footest.native
Finished, 10 targets (1 cached) in 00:00:00.
\end{verbatim}

Note that the \texttt{-cflags -warn-error,+26} is not indispensable but strongly recommended. Its
function will be explained in more detail in the more technical sections of this
documentation, but roughly it makes sure that if you write a test for \texttt{foo}, via
\texttt{(*\$T foo} for instance, then \texttt{foo} is \emph{actually} tested by each statement
-- the tests won't compile if not.

\textbf{Important note:} in order for this to work, \texttt{ocamlbuild} must know where to find
\texttt{foo.ml}; if \texttt{footest.ml} is not in the same directory, you must make provisions to that
effect. If \texttt{foo.ml} needs some specific flags in order to compile, they must also be
passed.


Now there only remains to run the tests:

\begin{verbatim}
$ ./footest.native
..FF
==============================================================================
Failure: qtest:0:foo:3:foo.ml:10

OUnit: foo.ml:10::>  foo 12 ( + ) [] = 12
------------------------------------------------------------------------------
==============================================================================
Failure: qtest:0:foo:2:foo.ml:9

OUnit: foo.ml:9::>  foo 1 ( * ) [4;5] = 20
------------------------------------------------------------------------------
Ran: 4 tests in: 0.00 seconds.
FAILED: Cases: 4 Tried: 4 Errors: 0 Failures: 2 Skip:0 Todo:0
\end{verbatim}

Oops, something's wrong... either the tests are incorrect or \texttt{foo} is. Finding and fixing
the problem is left as an exercise for the reader. When this is done, you get the expected

\begin{verbatim}
$ ./footest.native
....
Ran: 4 tests in: 0.00 seconds.
\end{verbatim}

\textbf{Tip:} those steps are easy to automate, for instance with a small shell script:


\begin{verbatim}
set -e # stop on first error
qtest -o footest.ml extract foo.ml
ocamlbuild -cflags -warn-error,+26 -use-ocamlfind -package oUnit footest.native
./footest.native
\end{verbatim}
\Overbatim